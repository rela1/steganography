\documentclass[times, utf8, seminar]{fer}
\usepackage{booktabs}
\usepackage{authblk}
\usepackage{mathtools}
\usepackage{amsmath}
\usepackage[binary-units=true]{siunitx}

%%floor symbol
\DeclarePairedDelimiter\floor{\lfloor}{\rfloor}

%%SI unit notation
\makeatletter
\providecommand\add@text{}
\newcommand\tagaddtext[1]{%
  \gdef\add@text{#1\gdef\add@text{}}}% 
\renewcommand\tagform@[1]{%
  \maketag@@@{\llap{\add@text\quad}(\ignorespaces#1\unskip\@@italiccorr)}%
}
\makeatother


\begin{document}

% TODO: Navedite naslov rada.
\title{Uporaba steganografije u revezibilnoj deidentifikaciji}

% TODO: Navedite vaše ime i prezime.
\author{	Katarina Matić,
	Hrvoje Backović,
	Marin Oršić,\\
	Dino Rakipović,
	Ivan Relić,
	Filip Reškov	}

% TODO: Navedite ime i prezime mentora.

\voditelj{Slobodan Ribarić}

\maketitle

\tableofcontents

\chapter{Projektni zadatak}
\section{Opis projektnog zadatka}
\section{Pregled i opis srodnih rješenja}
\section{Konceptualno rješenje zadatka}

\chapter{Postupak rješavanja zadataka}

\chapter{Ispitivanje rješenja}
Razvijeni sustav za steganografski postupak, kao i svi steganografski algoritmi, ograničen je količinom podataka koja se može upisati u neku sliku. Ova granica prvenstveno je određena veličinom slike, a određuju je i odabir kanala te najznačajniji bit do kojeg se upisuju podaci. Glavna ideja steganografskog algoritma jest upisati podatak u sliku bez velikog utjecaja na konačni izgled. Drugim riječima, izgled konačne slike uvjetovan je količinom podataka koji se u nju upisuju. Potrebno je pronaći dobre parametre steganografskog algoritma koji nude dobara kapacitet skrivenih podataka, a neznatno žrtvuju kvalitetu izvorne slike.
\par
Parametri algoritma koji su podešavani u ispitivanju su:
\begin{itemize}
\item Najznačajniji bit do kojeg se slijedno upisuje podatak
\item RGB komponente u koje će se upisivati podaci
\end{itemize}
Algoritam \textit{Least Significant Bit(LSB)} upisivanje sadržaja započinje s bitovima najnižeg značaja. Razlog tome leži u tome što se izmjenom bita najmanjeg značaja piksel najmanje mijenja. Praktični primjer bio bi kada bismo odlučili mijenjati samo B(\textit{blue}) komponentu i to samo najniži bit svakog bajta. Za svaki piksel slike(3 komponente, svaka po 1 bajt) dobije se jedan bit prostora za skrivanje podataka. Općenito, količina podataka koja se može upisati($n_{data}$), u ovisnosti o broju komponenti za upisivanje $n_{components}$ i broja najnižih bitova svake odabrane komponente za upisivanje $n_{bits}$ te veličini(broju piksela) $n_{pixels}$ slike je:
\begin{equation}
\label{num_bits}
n_{data} = \floor{\cfrac{n_{pixels} \cdot n_{components} \cdot n_{bits}}{8}}
\tagaddtext{[B]}
\end{equation}
\par
Udio veličine podataka $\eta$ koje je za dane parametre moguće upisati u sliku u odnosu na ukupnu veličinu slike je:
\begin{equation}
\label{eta}
\eta = \cfrac{n_{data}}{3 \cdot 8 \cdot n_{pixels}} = \cfrac{n_{components} \cdot n_{bits}}{24}
\end{equation}
Povećanjem $\eta$ vizualna razlika između izvorne slike i slike obrađene steganografskim algoritmom, u našem slučaju \textit{LSB}-om, povećava se. U nastavku slijede rezultati ispitivanja odnosa izvorne i obrađene slike u ovisnosti o parametrima $n_{components}$ i $n_{bits}$.
\section{Ispitna baza}
\section{Rezultati ispitivanja}
\section{Analiza rezultata}

\chapter{Opis programske implementacije rješenja}

\chapter{Zaključak}

\bibliography{literatura}
\bibliographystyle{fer}

\end{document}
